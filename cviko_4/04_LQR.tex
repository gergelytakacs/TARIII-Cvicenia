% cvicenie 1

\documentclass{beamer}

\mode<presentation>
{
  %\usetheme{Warsaw}
  %\usetheme{Singapore}
  %\usetheme{Szeged}
  \usetheme{Boadilla}

  \setbeamercovered{transparent}
  % or whatever (possibly just delete it)
}
\input newcomm
%\usepackage{slovak}
\usepackage{times}
\usepackage{psfrag}
\usepackage{graphics}
\usepackage{amsfonts}
\usepackage{amssymb}
\usepackage{amsmath}
\usepackage{theorem}
\usepackage{subfigure}
\usepackage{epsfig}
%\usepackage{natbib}
\usepackage{mathrsfs} %mathscr
\usepackage[cp1250]{inputenc}
\usepackage[T1]{fontenc}
\usepackage{listings}
\usepackage{multirow}
\usepackage{epstopdf}

\usepackage{amsmath}
\usepackage{amssymb}
\renewcommand{\vec}[1]{\boldsymbol{#1}} %I want bold vectors instead of arrows
\newcommand{\tab}[1]{\hspace{.1\textwidth}\rlap{#1}}

\title[TAR III.] % (optional, use only with long paper titles)
{Te�ria automatick�ho riadenia III.}
\subtitle{Cvi�enie IV, LQ riadenie}

\author[] % (optional, use only with lots of authors)
{G. Tak�cs, G. Batista E. Mikul\'{a}\v{s}}


\institute[UAMAI] % (optional, but mostly needed)
{
  �stav automatiz�cie, merania a aplikovanej informatiky\\
  Strojn�cka fakulta, Slovensk� technick� univerzita}
%  \and
%  \inst{2}%
%  Department of Theoretical Philosophy\\
%  University of Elsewhere}
% - Use the \inst command only if there are several affiliations.
% - Keep it simple, no one is interested in your street address.

\date[12.10.2015] % (optional, should be abbreviation of conference name)
{}
% - Either use conference name or its abbreviation.
% - Not really informative to the audience, more for people (including
%   yourself) who are reading the slides online

\subject{Automatiz�cia a riadenie}
% This is only inserted into the PDF information catalog. Can be left
% out.

%\begin{center}
%\includegraphics[height=2cm]{logo_stu_sjf_clr.eps} \\
%\end{center}

% If you have a file called "university-logo-filename.xxx", where xxx
% is a graphic format that can be processed by latex or pdflatex,
% resp., then you can add a logo as follows:

\pgfdeclareimage[height=1.5cm]{logo}{logo}
\logo{\pgfuseimage{logo}}



% Delete this, if you do not want the table of contents to pop up at
% the beginning of each subsection:
\AtBeginSubsection[]
{
  \begin{frame}<beamer>{Obsah}
    \tableofcontents[currentsection,currentsubsection]
  \end{frame}
}

\begin{document}
\setbeamertemplate{caption}{\raggedright\insertcaption\par}

\begin{frame}
  \titlepage
\end{frame}

\begin{frame}{�loha}
Pou�it�m identifikovan�ho modelu diskr�tneho modelu z predo�l�ho cvi�enia:
\begin{itemize}
\item Vytvorte LQ regul�tor
\item Overte jeho stabilitu
\item Urobte diskr�tnu simul�ciu otvorenej slu�ky a uzatvorenej s LQ riaden�m
\item Urobte diskr�tnu simul�ciu LQ riadenia so satur�ciou
\item Porovnajte odozvy v �asovej a frekven�nej oblasti
\end{itemize}
\end{frame}

\begin{frame}{LQ regul�tor}
\vspace{1cm}
Pozri:
\begin{itemize}
\item TARII - Week 10 - LQ Control.pdf
\end{itemize}
\end{frame}

\begin{frame}{Do zadania}
\begin{itemize}
\item MATLAB k�d v�ho zadania s koment�rom
\item zobrazenie kore�ov otvorenej a uzatvorenej slu�ky
\item porovnanie troch �asov�ch odoziev na impulz $ x_{0} = \begin{bmatrix} 0.01\\ 0 \end{bmatrix}$ \\ \small (vo�n� kmitanie, LQ riadenie, LQ riadenie so satur�ciou vstupu $\pm$ 90 V)
\item \normalsize porovnanie frekven�n�ch charakterist�k troch odoziev (\textbf{"periodogram"})
\item porovnanie �asov�ho priebehu vstupov regul�torov
\item grafy vykreslite tak aby boli v�etky rozdiely jasne vidite�n� a farebne odl�en�
\end{itemize}
\end{frame}

\end{document}